\documentclass[a4paper,11pt,oneside]{scrbook}
%
\usepackage[latin1]{inputenc}
\usepackage[T1]{fontenc}
\usepackage[pdftex]{color,graphicx}
\usepackage{fancyvrb}
\usepackage{geometry}
\usepackage{lastpage}
\usepackage{makeidx}
%
\geometry{a4paper,left=30mm,right=20mm,top=2.5cm,bottom=3cm}
\setlength{\headheight}{3.5em}
\setlength{\parindent}{0pt}
\setlength{\parskip}{1ex plus 0.5ex minus 0.2ex}
%
\title{Programming Style Guidelines\\ ArangoDB Edition}
\author{Dr. Frank Celler}
\date{Version 1.2.0}
%
\DefineVerbatimEnvironment%
  {code}{Verbatim}{commandchars=}
\DefineVerbatimEnvironment%
  {example}{Verbatim}{}
%
\newcommand{\guideline}[1]{{\subsection{#1}}}
\newcommand{\motivation}[1]{{\normalfont \itshape #1}}
\newcommand{\trfile}[1]{"#1"}
\newcommand{\trclass}[1]{\emph{#1}}
\newcommand{\trmethod}[1]{\emph{#1}}
\newcommand{\trtable}[1]{\emph{#1}}
\newcommand{\troption}[1]{\emph{#1}}
\newcommand{\trcode}[1]{{\normalfont \ttfamily #1}}
%
\begin{document}
\maketitle
\tableofcontents
%
%%%%%%%%%%%%%%%%%%%%%%%%%%%%%%%%%%%%%%%%%%%%%%%%%%%%%%%%%%%%%%%%%%%%%%%%%%%%%%%%
\chapter{Introduction}
%%%%%%%%%%%%%%%%%%%%%%%%%%%%%%%%%%%%%%%%%%%%%%%%%%%%%%%%%%%%%%%%%%%%%%%%%%%%%%%%

This document lists coding recommendations for the projects ArangoDB.  They are
based on C++ coding recommendations common in the C++ development community, on
established standards collected from a number of sources, individual experience,
local requirements/needs, as well as suggestions given in \ref{bib:1} -
\ref{bib:5} of chapter \ref{cha:references}.

There are several reasons for introducing a new guideline rather than just
referring to the ones above. The main reason is that these guides are far too
general in their scope and that more specific rules (especially naming rules)
need to be established. Also, the present guide has an annotated form that makes
it far easier to use during project code reviews than most other existing
guidelines. In addition, programming recommendations generally tend to mix style
issues with language technical issues in a somewhat confusing manner. The
present document does not contain any C++ technical recommendations at all, but
focuses mainly on programming style.

While a given development environment (IDE) can improve the readability of code
by access visibility, color coding, automatic formatting and so on, the
programmer should never rely on such features. Source code should always be
considered larger than the IDE it is developed within and should be written in a
way that maximize its readability independent of any IDE.

%%%%%%%%%%%%%%%%%%%%%%%%%%%%%%%%%%%%%%%%%%%%%%%%%%%%%%%%%%%%%%%%%%%%%%%%%%%%%%%%
\section{Layout of the Recommendations}
%%%%%%%%%%%%%%%%%%%%%%%%%%%%%%%%%%%%%%%%%%%%%%%%%%%%%%%%%%%%%%%%%%%%%%%%%%%%%%%%

The recommendations are grouped by topic and each recommendation is numbered to
make it easier to refer to during reviews. Layout of the recommendations is as
follows:

%~~~~~~~~~~~~~~~~~~~~~~~~~~~~~~~~~~~~~~~~~~~~~~~~~~~~~~~~~~~~~~~~~~~~~~~~~~~~~~~
\guideline{Guideline short description}
%~~~~~~~~~~~~~~~~~~~~~~~~~~~~~~~~~~~~~~~~~~~~~~~~~~~~~~~~~~~~~~~~~~~~~~~~~~~~~~~

Guideline longer description

\begin{code}
  Example if applicable
\end{code}
   
\motivation{
  Motivation, background and additional information.
}

The motivation section is important. Coding standards and guidelines
tend to start "religious wars", and it is important to state the
background for the recommendation.

%%%%%%%%%%%%%%%%%%%%%%%%%%%%%%%%%%%%%%%%%%%%%%%%%%%%%%%%%%%%%%%%%%%%%%%%%%%%%%%%
\section{Recommendation Importance}
%%%%%%%%%%%%%%%%%%%%%%%%%%%%%%%%%%%%%%%%%%%%%%%%%%%%%%%%%%%%%%%%%%%%%%%%%%%%%%%%

In the guideline sections the terms \emph{must}, \emph{should} and \emph{can}
have special meaning. A \emph{must} requirement must be followed, a
\emph{should} is a strong recommendation, and a \emph{can} is a general
guideline.

%%%%%%%%%%%%%%%%%%%%%%%%%%%%%%%%%%%%%%%%%%%%%%%%%%%%%%%%%%%%%%%%%%%%%%%%%%%%%%%%
\chapter{General Recommendations}
%%%%%%%%%%%%%%%%%%%%%%%%%%%%%%%%%%%%%%%%%%%%%%%%%%%%%%%%%%%%%%%%%%%%%%%%%%%%%%%%

\emph{Any} violation to the guide is allowed if it enhances readability.  The
main goal of the recommendation is to \emph{improve} readability and thereby the
understanding and the maintainability and general quality of the code. It is
impossible to cover all the specific cases in a general guide and the programmer
should be flexible.

However, as readability is subject to personal taste, any violation should be
discussed with the team. If a violation makes sense within a given context, a
corresponding recomemendation should be added to this document.

%%%%%%%%%%%%%%%%%%%%%%%%%%%%%%%%%%%%%%%%%%%%%%%%%%%%%%%%%%%%%%%%%%%%%%%%%%%%%%%%
\section{General Naming Conventions}
%%%%%%%%%%%%%%%%%%%%%%%%%%%%%%%%%%%%%%%%%%%%%%%%%%%%%%%%%%%%%%%%%%%%%%%%%%%%%%%%
\label{sec:general-naming-conventions}

%~~~~~~~~~~~~~~~~~~~~~~~~~~~~~~~~~~~~~~~~~~~~~~~~~~~~~~~~~~~~~~~~~~~~~~~~~~~~~~~
\guideline{Names must be chosen by forming an english description}
%~~~~~~~~~~~~~~~~~~~~~~~~~~~~~~~~~~~~~~~~~~~~~~~~~~~~~~~~~~~~~~~~~~~~~~~~~~~~~~~

Names must be chosen by forming an english description with "of", "in", "to",
"from" left out.

\begin{code}
  TypeLanguage  // NOT TypeOfLanguage or LanguageType
\end{code}

\motivation{
  Words like "of" are automatically added when reading.
}

%~~~~~~~~~~~~~~~~~~~~~~~~~~~~~~~~~~~~~~~~~~~~~~~~~~~~~~~~~~~~~~~~~~~~~~~~~~~~~~~
\guideline{Names representing types must be in upper camel case}
%~~~~~~~~~~~~~~~~~~~~~~~~~~~~~~~~~~~~~~~~~~~~~~~~~~~~~~~~~~~~~~~~~~~~~~~~~~~~~~~

Names representing types (Classes, Interfaces) must be in upper camel case
starting with upper case.

\begin{code}
  Line
  SavingsAccount
\end{code}

\motivation{
  Common practice in the C++ development community. 
}

Standard names are allowed to violated the above rule. Names used for instance
in the C++ STL do not always follow the above rule. In this case the established
names should be used. Examples are \trcode{size}, \trcode{length}, \trcode{at},
\trcode{c\_str}.

%~~~~~~~~~~~~~~~~~~~~~~~~~~~~~~~~~~~~~~~~~~~~~~~~~~~~~~~~~~~~~~~~~~~~~~~~~~~~~~~
\guideline{Variable names must be in lower camel case}
%~~~~~~~~~~~~~~~~~~~~~~~~~~~~~~~~~~~~~~~~~~~~~~~~~~~~~~~~~~~~~~~~~~~~~~~~~~~~~~~

Variable names must be in lower camel case starting with lower case.  Private
and protected member variables must start with a "\_".

\begin{code}
  line
  savingsAccount
\end{code}

\motivation{ 
  Common practice in the C++ development community. Makes variables easy to
  distinguish from types, and effectively resolves potential naming collision as
  in the declaration Line line.
}

%~~~~~~~~~~~~~~~~~~~~~~~~~~~~~~~~~~~~~~~~~~~~~~~~~~~~~~~~~~~~~~~~~~~~~~~~~~~~~~~
\guideline{Named constants must be all uppercase}
%~~~~~~~~~~~~~~~~~~~~~~~~~~~~~~~~~~~~~~~~~~~~~~~~~~~~~~~~~~~~~~~~~~~~~~~~~~~~~~~

Named constants (including enumeration values) must be all uppercase
using underscore to separate words.

\begin{code}
  MAX_ITERATIONS
  COLOR_RED
  PI 
\end{code}

\motivation{
  Common practice in the C++ development community. 
}

In general, the use of such constants should be minimized. In many cases
implementing the value as a method is a better choice:

\begin{code}
  int maxIterations () const {
    return 25;
  } 
\end{code}

This form is both easier to read, and it ensures a unified interface towards
class values.

%~~~~~~~~~~~~~~~~~~~~~~~~~~~~~~~~~~~~~~~~~~~~~~~~~~~~~~~~~~~~~~~~~~~~~~~~~~~~~~~
\guideline{Names representing methods or functions must be in lower camel case}
%~~~~~~~~~~~~~~~~~~~~~~~~~~~~~~~~~~~~~~~~~~~~~~~~~~~~~~~~~~~~~~~~~~~~~~~~~~~~~~~

Names representing methods or functions, which \emph{do some work}, must be
verbs and written in lower camel case starting with lower case. Names
representing methods or functions, which \emph{return something}, must be
substantives and written in lower camel case starting with lower case.

\begin{code}
  w = totalWidth();
  validateInput();
\end{code}

\motivation{ 
  Common practice in the C++ development community. This is identical to
  variable names, but functions in C++ are already distinguishable from
  variables by their specific form.

  Functions (methods returning something) should be named after what they return
  and procedures (void methods) after what they do.  This increases
  readability. Makes it clear what the unit should do and especially all the
  things it is not supposed to do. This again makes it easier to keep the code
  clean of side effects.
}

One exception to these rules are getter and setter - see below.
  
%~~~~~~~~~~~~~~~~~~~~~~~~~~~~~~~~~~~~~~~~~~~~~~~~~~~~~~~~~~~~~~~~~~~~~~~~~~~~~~~
\guideline{Names representing namespaces must be all lowercase}
%~~~~~~~~~~~~~~~~~~~~~~~~~~~~~~~~~~~~~~~~~~~~~~~~~~~~~~~~~~~~~~~~~~~~~~~~~~~~~~~

Names representing namespaces must be all lowercase using underscore to separate
words.

\begin{code}
  analyzer
  io_manager
\end{code}

\motivation{
  Common practice in the C++ development community.
}

%~~~~~~~~~~~~~~~~~~~~~~~~~~~~~~~~~~~~~~~~~~~~~~~~~~~~~~~~~~~~~~~~~~~~~~~~~~~~~~~
\guideline{Abbreviations and acronyms must not be uppercase when used as name}
%~~~~~~~~~~~~~~~~~~~~~~~~~~~~~~~~~~~~~~~~~~~~~~~~~~~~~~~~~~~~~~~~~~~~~~~~~~~~~~~

\begin{code}
  exportHtmlSource();   // NOT: exportHTMLSource();
  openDvdPlayer();      // NOT: openDVDPlayer(); 
\end{code}

\motivation{ 
  Using all uppercase for the base name will give conflicts with the naming
  conventions given above. A variable of this type whould have to be named dVD,
  hTML etc. which obviously is not very readable. Another problem is illustrated
  in the examples above; When the name is connected to another, the readbility
  is seriously reduced; the word following the abbreviation does not stand out
  as it should.
}

%~~~~~~~~~~~~~~~~~~~~~~~~~~~~~~~~~~~~~~~~~~~~~~~~~~~~~~~~~~~~~~~~~~~~~~~~~~~~~~~
\guideline{Generic variables must have the same name as their type}
%~~~~~~~~~~~~~~~~~~~~~~~~~~~~~~~~~~~~~~~~~~~~~~~~~~~~~~~~~~~~~~~~~~~~~~~~~~~~~~~

\begin{code}
  void setTopic (Topic *topic)  // NOT: void setTopic (Topic *value)
                                // NOT: void setTopic (Topic *aTopic) 
                                // NOT: void setTopic (Topic *x) 

  void connect (Database *database)  // NOT: void connect (Database *db)
                                     // NOT: void connect (Database *oracleDB) 
\end{code}

\motivation{ 
  Reduce complexity by reducing the number of terms and names used. Also makes
  it easy to deduce the type given a variable name only. If for some reason this
  convention doesn't seem to fit it is a strong indication that the type name is
  badly chosen.
}

%~~~~~~~~~~~~~~~~~~~~~~~~~~~~~~~~~~~~~~~~~~~~~~~~~~~~~~~~~~~~~~~~~~~~~~~~~~~~~~~
\guideline{Non-generic variables should have a role}
%~~~~~~~~~~~~~~~~~~~~~~~~~~~~~~~~~~~~~~~~~~~~~~~~~~~~~~~~~~~~~~~~~~~~~~~~~~~~~~~

Non-generic variables have a role. These variables can often be named by
combining role and type.

\begin{code}
  Point startingPoint, centerPoint;
  Name loginName;
\end{code}

%~~~~~~~~~~~~~~~~~~~~~~~~~~~~~~~~~~~~~~~~~~~~~~~~~~~~~~~~~~~~~~~~~~~~~~~~~~~~~~~
\guideline{All names should be written in English}
%~~~~~~~~~~~~~~~~~~~~~~~~~~~~~~~~~~~~~~~~~~~~~~~~~~~~~~~~~~~~~~~~~~~~~~~~~~~~~~~

\begin{code}
  FileName;     // NOT: filNavn 
\end{code}

\motivation{
  English is the preferred language for international development.
}

%~~~~~~~~~~~~~~~~~~~~~~~~~~~~~~~~~~~~~~~~~~~~~~~~~~~~~~~~~~~~~~~~~~~~~~~~~~~~~~~
\guideline{The length of a name should correspond to the scope}
%~~~~~~~~~~~~~~~~~~~~~~~~~~~~~~~~~~~~~~~~~~~~~~~~~~~~~~~~~~~~~~~~~~~~~~~~~~~~~~~

Variables with a large scope should have long names, variables with a small
scope can have short names.

\motivation{ 
  Scratch variables used for temporary storage or indices are best kept short. A
  programmer reading such variables should be able to assume that its value is
  not used outside a few lines of code. Common scratch variables for integers
  are i, j, k, m, n and for characters c and d.
}

%~~~~~~~~~~~~~~~~~~~~~~~~~~~~~~~~~~~~~~~~~~~~~~~~~~~~~~~~~~~~~~~~~~~~~~~~~~~~~~~
\guideline{The maximal length of a name must be 40}
%~~~~~~~~~~~~~~~~~~~~~~~~~~~~~~~~~~~~~~~~~~~~~~~~~~~~~~~~~~~~~~~~~~~~~~~~~~~~~~~

\motivation{ 
  Very long names make the program harder to read because more line breaks are
  required. If the very long name is required to describe a method or variable
  it could be an indication that this method or variable is to complex and that
  the code should be refactored.
}

%~~~~~~~~~~~~~~~~~~~~~~~~~~~~~~~~~~~~~~~~~~~~~~~~~~~~~~~~~~~~~~~~~~~~~~~~~~~~~~~
\guideline{The name of the object should be avoided in a method name}
%~~~~~~~~~~~~~~~~~~~~~~~~~~~~~~~~~~~~~~~~~~~~~~~~~~~~~~~~~~~~~~~~~~~~~~~~~~~~~~~

The name of the object is implicit, and should be avoided in a method name. 

\begin{code}
  line.getLength();  // NOT: line.getLineLength(); 
\end{code}

\motivation{ 
  The latter seems natural in the class declaration, but proves superfluous in
  use, as shown in the example.
}
  
%%%%%%%%%%%%%%%%%%%%%%%%%%%%%%%%%%%%%%%%%%%%%%%%%%%%%%%%%%%%%%%%%%%%%%%%%%%%%%%%
\section{Specific Naming Conventions}
%%%%%%%%%%%%%%%%%%%%%%%%%%%%%%%%%%%%%%%%%%%%%%%%%%%%%%%%%%%%%%%%%%%%%%%%%%%%%%%%

%~~~~~~~~~~~~~~~~~~~~~~~~~~~~~~~~~~~~~~~~~~~~~~~~~~~~~~~~~~~~~~~~~~~~~~~~~~~~~~~
\guideline{Getters and setters}
%~~~~~~~~~~~~~~~~~~~~~~~~~~~~~~~~~~~~~~~~~~~~~~~~~~~~~~~~~~~~~~~~~~~~~~~~~~~~~~~

The terms \trcode{get}/\trcode{set} must be used where an attribute is accessed
directly

\begin{code}
  employee.getName();
  matrix.getElement(2, 4);
  employee.setName(name);
  matrix.setElement(2, 4, value); 
\end{code}

\motivation{
  In Java this convention has become more or less standard.
}

Note that this is only used when accessing a member variable directly. If any
computation is involved the above rule does not apply and section
\ref{sec:general-naming-conventions} should be followed. A getter or setter must
only consists of a few line of code.

The terms \trcode{is}/\trcode{set} must be used where a boolean attribute is
accessed directly.

\begin{code}
  matrix.isDense();
\end{code}

\motivation{ 
  In Java this convention has become more or less standard. There are exceptions
  to this rule in C++. The STL for example defines \trcode{empty} instead
  \trcode{isEmpty}.
}

If definition a setter \trcode{setSomething} use \trcode{newSomething} as
variable name for the new value.

%~~~~~~~~~~~~~~~~~~~~~~~~~~~~~~~~~~~~~~~~~~~~~~~~~~~~~~~~~~~~~~~~~~~~~~~~~~~~~~~
\guideline{Should use \trcode{compute} for methods which compute and store}
%~~~~~~~~~~~~~~~~~~~~~~~~~~~~~~~~~~~~~~~~~~~~~~~~~~~~~~~~~~~~~~~~~~~~~~~~~~~~~~~

The term \trcode{compute} should be used in methods where something is computed
and stored within the object.

\begin{code}
  valueSet->computeAverage();
  matrix->computeInverse(); 
\end{code}

\motivation{ 
  Give the reader the immediate clue that this is a potential time consuming
  operation, and if used repeatedly, he might consider caching the
  result. Consistent use of the term enhances readability.
}

%~~~~~~~~~~~~~~~~~~~~~~~~~~~~~~~~~~~~~~~~~~~~~~~~~~~~~~~~~~~~~~~~~~~~~~~~~~~~~~~
\guideline{Should use \trcode{find} and \trcode{lookup} for look ups}
%~~~~~~~~~~~~~~~~~~~~~~~~~~~~~~~~~~~~~~~~~~~~~~~~~~~~~~~~~~~~~~~~~~~~~~~~~~~~~~~

The terms \trcode{find} and \trcode{lookup} should be used in methods where
something is looked up.

\begin{code}
  vertex.findNearestVertex();
  matrix.findMinElement();
  vertex.lookupVertex(vertexId);
\end{code}

\motivation{
  Give the reader the immediate clue that this is a simple look up method with a
  minimum of computations involved. \trcode{lookup} should be used when no
  computation or minimal computation is done and no error is raised when the
  element is not found. \trcode{find} should be used when computations are
  required; it is allowed to create missing elements or to raise an
  error. Consistent use of the terms enhances readability.
}

%~~~~~~~~~~~~~~~~~~~~~~~~~~~~~~~~~~~~~~~~~~~~~~~~~~~~~~~~~~~~~~~~~~~~~~~~~~~~~~~
\guideline{Should use \trcode{initialize} for initialisation}
%~~~~~~~~~~~~~~~~~~~~~~~~~~~~~~~~~~~~~~~~~~~~~~~~~~~~~~~~~~~~~~~~~~~~~~~~~~~~~~~

The term \trcode{initialize} should be used where an object or a concept is
established.

\begin{code}
  printer.initializeFontSet(); 
\end{code}

\motivation{
  The American initialize should be preferred over the British initialise.  The
  abbreviation \trcode{init} should be avoided.
}

%~~~~~~~~~~~~~~~~~~~~~~~~~~~~~~~~~~~~~~~~~~~~~~~~~~~~~~~~~~~~~~~~~~~~~~~~~~~~~~~
\guideline{Should use GUI component type name as suffix}
%~~~~~~~~~~~~~~~~~~~~~~~~~~~~~~~~~~~~~~~~~~~~~~~~~~~~~~~~~~~~~~~~~~~~~~~~~~~~~~~

Variables representing GUI components should be suffixed by the component type
name.

\begin{code}
  mainWindow, propertiesDialog, widthScale, loginText, leftScrollbar,
  mainForm, fileMenu, minLabel, exitButton, yesToggle
\end{code}

\motivation{ 
  Enhances readability since the name gives the user an immediate clue of the
  type of the variable and thereby the objects resources.
}

%~~~~~~~~~~~~~~~~~~~~~~~~~~~~~~~~~~~~~~~~~~~~~~~~~~~~~~~~~~~~~~~~~~~~~~~~~~~~~~~
\guideline{Can use \trcode{List} suffix for lists}
%~~~~~~~~~~~~~~~~~~~~~~~~~~~~~~~~~~~~~~~~~~~~~~~~~~~~~~~~~~~~~~~~~~~~~~~~~~~~~~~

The suffix \trcode{List} can be used on names representing a list of objects. 

\begin{code}
  vertex (one vertex)
  vertexList (a list of vertices) 
\end{code}

\motivation{ 
  Enhances readability since the name gives the user an immediate clue of the
  type of the variable and the operations that can be performed on the
  object. Simply using the plural form of the base class name for a list
  (matrixElement (one matrix element), matrixElements (list of matrix elements))
  shoul be avoided since the two only differ in a single character and are
  thereby difficult to distinguish.
}

A list in this context is the compound data type that can be traversed
backwards, forwards, etc. (typically an STL vector). A plain array is
simpler. The suffix \trcode{Array} can be used to denote an array of objects.
  
%~~~~~~~~~~~~~~~~~~~~~~~~~~~~~~~~~~~~~~~~~~~~~~~~~~~~~~~~~~~~~~~~~~~~~~~~~~~~~~~
\guideline{Should use prefix \trcode{n} or \trcode{number}}
%~~~~~~~~~~~~~~~~~~~~~~~~~~~~~~~~~~~~~~~~~~~~~~~~~~~~~~~~~~~~~~~~~~~~~~~~~~~~~~~

The prefix \trcode{n} or \trcode{number} should be used for variables
representing a number of objects.

\begin{code}
  nPoints, numberLines 
\end{code}

\motivation{ 
  The notation is taken from mathematics where it is an established convention
  for indicating a number of objects.
}

%~~~~~~~~~~~~~~~~~~~~~~~~~~~~~~~~~~~~~~~~~~~~~~~~~~~~~~~~~~~~~~~~~~~~~~~~~~~~~~~
\guideline{Should use suffix \trcode{Id} for identifier}
%~~~~~~~~~~~~~~~~~~~~~~~~~~~~~~~~~~~~~~~~~~~~~~~~~~~~~~~~~~~~~~~~~~~~~~~~~~~~~~~

The suffix \trcode{Id} should be used for variables representing an entity
  number.

\begin{code}
  tableId, employeeId 
\end{code}

\motivation{ 
  The notation is taken from mathematics where it is an established convention
  for indicating an entity number.
}

%~~~~~~~~~~~~~~~~~~~~~~~~~~~~~~~~~~~~~~~~~~~~~~~~~~~~~~~~~~~~~~~~~~~~~~~~~~~~~~~
\guideline{Should use mathematical iterators names}
%~~~~~~~~~~~~~~~~~~~~~~~~~~~~~~~~~~~~~~~~~~~~~~~~~~~~~~~~~~~~~~~~~~~~~~~~~~~~~~~

Iterator variables should be called \trcode{i}, \trcode{j}, \trcode{k} etc
or \trcode{iter}. 

\begin{code}
  for (int i = 0; i < nTables); i++) {
    ...
  }

  vector<MyClass>::iterator iter;

  for (iter = list.begin();  iter != list.end();  ++iter) { 
    Element element = *iter;
    ... 
  } 
\end{code}

\motivation{ 
  The notation is taken from mathematics where it is an established convention
  for indicating iterators. \trcode{iter} should be used for STL iterators.
}

An elegant alternative is to prefix such variables with an i or iter: iTable,
iterEmployee. This effectively makes them named iterators.

%~~~~~~~~~~~~~~~~~~~~~~~~~~~~~~~~~~~~~~~~~~~~~~~~~~~~~~~~~~~~~~~~~~~~~~~~~~~~~~~
\guideline{The prefix \trcode{is} should be used for boolean variables and methods}
%~~~~~~~~~~~~~~~~~~~~~~~~~~~~~~~~~~~~~~~~~~~~~~~~~~~~~~~~~~~~~~~~~~~~~~~~~~~~~~~

\begin{code}
  isSet, isVisible, isFinished, isFound, isOpen 
\end{code}

\motivation{
  Common practice in the C++ development community and partially enforced in Java. 
}

Using the is prefix solves a common problem of choosing bad boolean names like
status or flag. isStatus or isFlag simply doesn't fit, and the programmer is
forced to choose more meaningful names. There are a few alternatives to the
\trcode{is} prefix that fits better in some situations. These are the
\trcode{has}, \trcode{can} and \trcode{should} prefixes:

\begin{code}
  bool hasLicense(); 
  bool canEvaluate(); 
  bool shouldSort(); 
\end{code}

%~~~~~~~~~~~~~~~~~~~~~~~~~~~~~~~~~~~~~~~~~~~~~~~~~~~~~~~~~~~~~~~~~~~~~~~~~~~~~~~
\guideline{Complement names must be used for complement operations}
%~~~~~~~~~~~~~~~~~~~~~~~~~~~~~~~~~~~~~~~~~~~~~~~~~~~~~~~~~~~~~~~~~~~~~~~~~~~~~~~

\begin{code}
  get/set
  add/remove
  create/destroy
  start/stop
  insert/delete
  increment/decrement
  old/new
  begin/end
  first/last
  up/down
  min/max
  next/previous
  open/close
  show/hide
  suspend/resume
\end{code}

\motivation{
  Reduce complexity by symmetry.
}

%~~~~~~~~~~~~~~~~~~~~~~~~~~~~~~~~~~~~~~~~~~~~~~~~~~~~~~~~~~~~~~~~~~~~~~~~~~~~~~~
\guideline{Abbreviations in names must be avoided}
%~~~~~~~~~~~~~~~~~~~~~~~~~~~~~~~~~~~~~~~~~~~~~~~~~~~~~~~~~~~~~~~~~~~~~~~~~~~~~~~

\begin{code}
  computeAverage();     // NOT: compAvg(); 
\end{code}

There are two types of words to consider. First are the common words listed in a
language dictionary. These must never be abbreviated. Never write:

\begin{itemize}
  \item
    cmd instead of command

  \item
    cp instead of copy

  \item
    pt instead of point

  \item
    comp instead of compute

  \item
    init instead of initialize
\end{itemize}

Then there are domain specific phrases that are more naturally known through
their abbreviations/acronym. These phrases must be kept abbreviated. Never
write:

\begin{itemize}
  \item
    HypertextMarkupLanguage instead of html

  \item
    CentralProcessingUnit instead of cpu

  \item
    PriceEarningRatio instead of pe
\end{itemize}

%~~~~~~~~~~~~~~~~~~~~~~~~~~~~~~~~~~~~~~~~~~~~~~~~~~~~~~~~~~~~~~~~~~~~~~~~~~~~~~~
\guideline{Naming pointers specifically should be avoided}
%~~~~~~~~~~~~~~~~~~~~~~~~~~~~~~~~~~~~~~~~~~~~~~~~~~~~~~~~~~~~~~~~~~~~~~~~~~~~~~~

\begin{code}
  Line * line;  // NOT: Line * pLine; or Line * linePtr; etc. 
\end{code}

Many variables in a C/C++ environment are pointers, so a convention like this is
almost impossible to follow. Also objects in C++ are often oblique types where
the specific implementation should be ignored by the programmer. Only when the
actual type of an object is of special significance, the name should emphasize
the type.

%~~~~~~~~~~~~~~~~~~~~~~~~~~~~~~~~~~~~~~~~~~~~~~~~~~~~~~~~~~~~~~~~~~~~~~~~~~~~~~~
\guideline{Negated boolean variable names must be avoided}
%~~~~~~~~~~~~~~~~~~~~~~~~~~~~~~~~~~~~~~~~~~~~~~~~~~~~~~~~~~~~~~~~~~~~~~~~~~~~~~~

\begin{code}
  bool isError; // NOT: isNoError
  bool isFound; // NOT: isNotFound 
\end{code}

The problem arises when such a name is used in conjunction with the logical
negation operator as this results in a double negative. It is not immediately
apparent what \trcode{!isNotFound} means.

%~~~~~~~~~~~~~~~~~~~~~~~~~~~~~~~~~~~~~~~~~~~~~~~~~~~~~~~~~~~~~~~~~~~~~~~~~~~~~~~
\guideline{Enumeration constants can be prefixed by a common type name}
%~~~~~~~~~~~~~~~~~~~~~~~~~~~~~~~~~~~~~~~~~~~~~~~~~~~~~~~~~~~~~~~~~~~~~~~~~~~~~~~

\begin{code}
  enum color_e {
    COLOR_RED,
    COLOR_GREEN,
    COLOR_BLUE
  }; 
\end{code}

\motivation{
  This gives additional information of where the declaration can be found, which
  constants belongs together, and what concept the constants represent. An
  alternative approach is to always refer to the constants through their common
  type: \trcode{Color::RED}, \trcode{Airline::AIR\_FRANCE}, etc.
}

%%%%%%%%%%%%%%%%%%%%%%%%%%%%%%%%%%%%%%%%%%%%%%%%%%%%%%%%%%%%%%%%%%%%%%%%%%%%%%%%
\section{C++ Naming Conventions}
%%%%%%%%%%%%%%%%%%%%%%%%%%%%%%%%%%%%%%%%%%%%%%%%%%%%%%%%%%%%%%%%%%%%%%%%%%%%%%%%

%~~~~~~~~~~~~~~~~~~~~~~~~~~~~~~~~~~~~~~~~~~~~~~~~~~~~~~~~~~~~~~~~~~~~~~~~~~~~~~~
\guideline{Within a given module, class names must be unique}
%~~~~~~~~~~~~~~~~~~~~~~~~~~~~~~~~~~~~~~~~~~~~~~~~~~~~~~~~~~~~~~~~~~~~~~~~~~~~~~~
\label{subsec:unique-class-name}

\motivation{
  If class names are unique it is much easier to understand the code because
  mix-ups are avoided.
}

%~~~~~~~~~~~~~~~~~~~~~~~~~~~~~~~~~~~~~~~~~~~~~~~~~~~~~~~~~~~~~~~~~~~~~~~~~~~~~~~
\guideline{Exception classes should be suffixed with \trcode{Exception} or \trcode{Error}}
%~~~~~~~~~~~~~~~~~~~~~~~~~~~~~~~~~~~~~~~~~~~~~~~~~~~~~~~~~~~~~~~~~~~~~~~~~~~~~~~

\begin{code}
  class AccessException {
    ...
  }
\end{code}

\motivation{ 
  Exception classes are really not part of the main design of the program, and
  naming them like this makes them stand out relative to the other classes.
}

%~~~~~~~~~~~~~~~~~~~~~~~~~~~~~~~~~~~~~~~~~~~~~~~~~~~~~~~~~~~~~~~~~~~~~~~~~~~~~~~
\guideline{Enumerations should be lowercase followed by \trcode{\_e}}
%~~~~~~~~~~~~~~~~~~~~~~~~~~~~~~~~~~~~~~~~~~~~~~~~~~~~~~~~~~~~~~~~~~~~~~~~~~~~~~~

\begin{code}
  enum color_e {
    ..
  };
\end{code}

%~~~~~~~~~~~~~~~~~~~~~~~~~~~~~~~~~~~~~~~~~~~~~~~~~~~~~~~~~~~~~~~~~~~~~~~~~~~~~~~
\guideline{Template names should are short and must be uppercase}
%~~~~~~~~~~~~~~~~~~~~~~~~~~~~~~~~~~~~~~~~~~~~~~~~~~~~~~~~~~~~~~~~~~~~~~~~~~~~~~~

Names representing template types must be uppercase.

Names representing generic template types should be a single uppercase
letter. Two uppercase letters at most. Names representing specific template
types should be named after that type.

\begin{code}
  template<typename T> ... 
  template<typename C, typename D> ... 
  template<typename C, class VERTEX> ...
\end{code}

\motivation{ 
  Common practice in the C++ development community. This makes template names
  stand out relative to all other names used.
}

%~~~~~~~~~~~~~~~~~~~~~~~~~~~~~~~~~~~~~~~~~~~~~~~~~~~~~~~~~~~~~~~~~~~~~~~~~~~~~~~
\guideline{Global variables should always be referred to using the \trcode{::} operator}
%~~~~~~~~~~~~~~~~~~~~~~~~~~~~~~~~~~~~~~~~~~~~~~~~~~~~~~~~~~~~~~~~~~~~~~~~~~~~~~~

\begin{code}
  ::mainWindow.open()
  ::applicationContext.getName() 
\end{code}

\motivation{ 
  In general, the use of global variables should be avoided. Consider using
  singleton objects instead.
}

%%%%%%%%%%%%%%%%%%%%%%%%%%%%%%%%%%%%%%%%%%%%%%%%%%%%%%%%%%%%%%%%%%%%%%%%%%%%%%%%
\chapter{Files}
%%%%%%%%%%%%%%%%%%%%%%%%%%%%%%%%%%%%%%%%%%%%%%%%%%%%%%%%%%%%%%%%%%%%%%%%%%%%%%%%

%%%%%%%%%%%%%%%%%%%%%%%%%%%%%%%%%%%%%%%%%%%%%%%%%%%%%%%%%%%%%%%%%%%%%%%%%%%%%%%%
\section{Source Files}
%%%%%%%%%%%%%%%%%%%%%%%%%%%%%%%%%%%%%%%%%%%%%%%%%%%%%%%%%%%%%%%%%%%%%%%%%%%%%%%%

%~~~~~~~~~~~~~~~~~~~~~~~~~~~~~~~~~~~~~~~~~~~~~~~~~~~~~~~~~~~~~~~~~~~~~~~~~~~~~~~
\guideline{Must use \trfile{.h} and \trfile{.cpp} extensions}
%~~~~~~~~~~~~~~~~~~~~~~~~~~~~~~~~~~~~~~~~~~~~~~~~~~~~~~~~~~~~~~~~~~~~~~~~~~~~~~~

C++ header files must have the extension \trfile{.h}. Source files must the
extension \trfile{.cpp}. The extensions \trfile{.C}, \trfile{.cc} or
\trfile{.c++} are not allowed.

\begin{code}
  MyClass.cpp, MyClass.h 
\end{code}

\motivation{
  These are all accepted C++ standards for file extension.
}

%~~~~~~~~~~~~~~~~~~~~~~~~~~~~~~~~~~~~~~~~~~~~~~~~~~~~~~~~~~~~~~~~~~~~~~~~~~~~~~~
\guideline{Must use two files per class for public classes}
%~~~~~~~~~~~~~~~~~~~~~~~~~~~~~~~~~~~~~~~~~~~~~~~~~~~~~~~~~~~~~~~~~~~~~~~~~~~~~~~
\label{subsec:two-files-per-class}

A class must be declared in a header file and implemented in a source file where
the name of the files match the name of the class. Classes which are local to
computation and have a file context, can be defined in the source file in an
anonymous namespace.

The files must have the same name as the class including case. A class
\trcode{MyHtml} must be declared in \trfile{MyHtml.h"} and implemented in
\trfile{MyHtml.cpp}.

\begin{code}
  MyClass.h, MyClass.cpp 
\end{code}

\motivation{ 
  Makes it easy to find the associated files of a given class. This convention
  is enforced in Java and has become very successful as such.
}

%~~~~~~~~~~~~~~~~~~~~~~~~~~~~~~~~~~~~~~~~~~~~~~~~~~~~~~~~~~~~~~~~~~~~~~~~~~~~~~~
\guideline{All computations must reside in source files}
%~~~~~~~~~~~~~~~~~~~~~~~~~~~~~~~~~~~~~~~~~~~~~~~~~~~~~~~~~~~~~~~~~~~~~~~~~~~~~~~

\begin{code}
  class MyClass { 
    public: 
      int getValue () {return value;} // YES
      int computeSomething () {...computation...} // NO! 

    private: 
      int value; 
  } 
\end{code}

\motivation{ 
  The header files must declare an interface of a class, the source file must
  implement it. When looking for an implementation, the programmer should always
  know that it is found in the source file. The obvious exception to this rule
  are of course inline functions and templates that must be defined in the
  header file. Trivial getter and setter are another exception.
}

%~~~~~~~~~~~~~~~~~~~~~~~~~~~~~~~~~~~~~~~~~~~~~~~~~~~~~~~~~~~~~~~~~~~~~~~~~~~~~~~
\guideline{File content should be kept within 80 columns}
%~~~~~~~~~~~~~~~~~~~~~~~~~~~~~~~~~~~~~~~~~~~~~~~~~~~~~~~~~~~~~~~~~~~~~~~~~~~~~~~

80 columns is a common dimension for editors, terminal emulators, printers and
debuggers, and files that are shared between several people should keep within
these constraints. It improves readability when unintentional line breaks are
avoided when passing a file between programmers.  Special characters like TAB
and page break must be avoided.

These characters are bound to cause problem for editors, printers, terminal
emulators or debuggers when used in a multi-programmer, multi-platform
environment.  The incompleteness of split lines must be made obvious.

\begin{code}
  totalSum = a + b + c
           + d + e;

  function (param1, param2,
           param3);

  setText ("Long line split"
           "into two parts.");

  for (tableNo = 0; tableNo < nTables;
       tableNo += tableStep) 
\end{code}

Split lines occurs when a statement exceed the 80 column limit given above. It
is difficult to give rigid rules for how lines should be split, but the examples
above should give a general hint. In general:

\begin{itemize}
  \item
    Break after a comma.

  \item
    Break after an operator. 

  \item
    Align the new line with the beginning of the expression on the previous line. 
\end{itemize}

%%%%%%%%%%%%%%%%%%%%%%%%%%%%%%%%%%%%%%%%%%%%%%%%%%%%%%%%%%%%%%%%%%%%%%%%%%%%%%%%
\section{Include Files and Include Statements}
%%%%%%%%%%%%%%%%%%%%%%%%%%%%%%%%%%%%%%%%%%%%%%%%%%%%%%%%%%%%%%%%%%%%%%%%%%%%%%%%

%~~~~~~~~~~~~~~~~~~~~~~~~~~~~~~~~~~~~~~~~~~~~~~~~~~~~~~~~~~~~~~~~~~~~~~~~~~~~~~~
\guideline{Must use include guards}
%~~~~~~~~~~~~~~~~~~~~~~~~~~~~~~~~~~~~~~~~~~~~~~~~~~~~~~~~~~~~~~~~~~~~~~~~~~~~~~~
\label{guideline:define-guard}

Header files must include a construction that prevents multiple inclusion. The
convention is an all uppercase construction of the module/directory name, the
file name and the \trcode{h} suffix.

Obeying \ref{subsec:two-files-per-class} and \ref{subsec:unique-class-name}
guarantees that the input guards is unique.

\begin{code}
  #ifndef MODULE_FILENAME_H
  #define MODULE_FILENAME_H
  ...
  #endif 
\end{code}

\motivation{
  The construction is to avoid compilation errors. The name convention is common
  practice. The construction should appear in the top of the file (before the
  file header) so file parsing is aborted immediately and compilation time is
  reduced.
}

%~~~~~~~~~~~~~~~~~~~~~~~~~~~~~~~~~~~~~~~~~~~~~~~~~~~~~~~~~~~~~~~~~~~~~~~~~~~~~~~
\guideline{Include statements should be sorted and grouped}
%~~~~~~~~~~~~~~~~~~~~~~~~~~~~~~~~~~~~~~~~~~~~~~~~~~~~~~~~~~~~~~~~~~~~~~~~~~~~~~~

Sorted by their hierarchical position in the system with low level files
included first. Leave an empty line between groups of include statements.

\begin{code}
  #include <fstream>
  #include <iomanip>

  #include <Xm/Xm.h>
  #include <Xm/ToggleB.h>

  #include "ui/PropertiesDialog.h"
  #include "ui/MainWindow.h" 
\end{code}

\motivation{
  In addition to show the reader the individual include files, it also give an
  immediate clue about the modules that are involved. Include file paths must
  never be absolute. Compiler directives should instead be used to indicate root
  directories for includes.
}

%~~~~~~~~~~~~~~~~~~~~~~~~~~~~~~~~~~~~~~~~~~~~~~~~~~~~~~~~~~~~~~~~~~~~~~~~~~~~~~~
\guideline{Include statements must be located at the top of a file only}
%~~~~~~~~~~~~~~~~~~~~~~~~~~~~~~~~~~~~~~~~~~~~~~~~~~~~~~~~~~~~~~~~~~~~~~~~~~~~~~~

\motivation{ 
  Common practice. Avoid unwanted compilation side effects
  by "hidden" include statements deep into a source file.
}

%%%%%%%%%%%%%%%%%%%%%%%%%%%%%%%%%%%%%%%%%%%%%%%%%%%%%%%%%%%%%%%%%%%%%%%%%%%%%%%%
\section{File Structure}
%%%%%%%%%%%%%%%%%%%%%%%%%%%%%%%%%%%%%%%%%%%%%%%%%%%%%%%%%%%%%%%%%%%%%%%%%%%%%%%%

%~~~~~~~~~~~~~~~~~~~~~~~~~~~~~~~~~~~~~~~~~~~~~~~~~~~~~~~~~~~~~~~~~~~~~~~~~~~~~~~
\guideline{Header file structure}
%~~~~~~~~~~~~~~~~~~~~~~~~~~~~~~~~~~~~~~~~~~~~~~~~~~~~~~~~~~~~~~~~~~~~~~~~~~~~~~~

The parts of a class must be sorted public, protected and private. All
sections must be identified explicitly. Not applicable sections should
be left out.  The ordering is "most public first" so people who only
wish to use the class can stop reading when they reach the
protected/private sections.

Use the following ordering within a header file.

\begin{enumerate}
  \item
    header including copyright, author, and date

  \item
    define-guard, see \ref{guideline:define-guard}

  \item
    include \trfile{Common.h} or header-file of base classes

  \item
    include system C headers

  \item
    include system C++ headers

  \item
    include project C++ headers, use forward declarations where possible

  \item
    forward declarations for C++ classes and structs

  \item
    class declaration

    \begin{enumerate}
      \item
        local classes and enumerations

      \item
        static constant variables

      \item
        static functions

      \item
        static variables

      \item
        constructors and destructors

      \item
        public methods

      \item
        public variables

      \item
        protected methods

      \item
        protected variables

      \item
        private methods

      \item
        private variables
    \end{enumerate}
\end{enumerate}

%~~~~~~~~~~~~~~~~~~~~~~~~~~~~~~~~~~~~~~~~~~~~~~~~~~~~~~~~~~~~~~~~~~~~~~~~~~~~~~~
\guideline{Source file structure}
%~~~~~~~~~~~~~~~~~~~~~~~~~~~~~~~~~~~~~~~~~~~~~~~~~~~~~~~~~~~~~~~~~~~~~~~~~~~~~~~

The parts of a class must be sorted public, protected and private. All
sections must be identified explicitly. Not applicable sections should
be left out.  The ordering is "most public first" so people who only
wish to use the class can stop reading when they reach the
protected/private sections.

Use the following ordering within a source file.

\begin{enumerate}
  \item
    header including copyright, author, and date

  \item
    corresponding header file

  \item
    include system C headers

  \item
    include system C++ headers

  \item
    include project C++ headers

  \item
    class implementation

    \begin{enumerate}
      \item
        auxillary function hidden in an anonymous namespace

      \item
        static constant variables

      \item
        static functions

      \item
        constructors and destructors

      \item
        public methods

      \item
        protected methods

      \item
        private methods
    \end{enumerate}
\end{enumerate}

%%%%%%%%%%%%%%%%%%%%%%%%%%%%%%%%%%%%%%%%%%%%%%%%%%%%%%%%%%%%%%%%%%%%%%%%%%%%%%%%
\chapter{C++ Statements}
%%%%%%%%%%%%%%%%%%%%%%%%%%%%%%%%%%%%%%%%%%%%%%%%%%%%%%%%%%%%%%%%%%%%%%%%%%%%%%%%

%%%%%%%%%%%%%%%%%%%%%%%%%%%%%%%%%%%%%%%%%%%%%%%%%%%%%%%%%%%%%%%%%%%%%%%%%%%%%%%%
\section{Namespaces}
%%%%%%%%%%%%%%%%%%%%%%%%%%%%%%%%%%%%%%%%%%%%%%%%%%%%%%%%%%%%%%%%%%%%%%%%%%%%%%%%

%~~~~~~~~~~~~~~~~~~~~~~~~~~~~~~~~~~~~~~~~~~~~~~~~~~~~~~~~~~~~~~~~~~~~~~~~~~~~~~~
\guideline{Namespaces must not be included globally in the header}
%~~~~~~~~~~~~~~~~~~~~~~~~~~~~~~~~~~~~~~~~~~~~~~~~~~~~~~~~~~~~~~~~~~~~~~~~~~~~~~~

A namespace must be included by \trcode{using} either within the source file
or within another namespace. It must not be included globally in a header file.

\motivation{
  Avoids conflicts with other libraries.
}

%%%%%%%%%%%%%%%%%%%%%%%%%%%%%%%%%%%%%%%%%%%%%%%%%%%%%%%%%%%%%%%%%%%%%%%%%%%%%%%%
\section{Types}
%%%%%%%%%%%%%%%%%%%%%%%%%%%%%%%%%%%%%%%%%%%%%%%%%%%%%%%%%%%%%%%%%%%%%%%%%%%%%%%%

%~~~~~~~~~~~~~~~~~~~~~~~~~~~~~~~~~~~~~~~~~~~~~~~~~~~~~~~~~~~~~~~~~~~~~~~~~~~~~~~
\guideline{Types that are local to one file only can be declared inside that file}
%~~~~~~~~~~~~~~~~~~~~~~~~~~~~~~~~~~~~~~~~~~~~~~~~~~~~~~~~~~~~~~~~~~~~~~~~~~~~~~~

Classes which are local to computation and have a file context, can
be defined in the source file in an anonymous namespace.

\motivation{
  Enforces information hiding. 
}

%~~~~~~~~~~~~~~~~~~~~~~~~~~~~~~~~~~~~~~~~~~~~~~~~~~~~~~~~~~~~~~~~~~~~~~~~~~~~~~~
\guideline{Type conversions must always be done explicitly}
%~~~~~~~~~~~~~~~~~~~~~~~~~~~~~~~~~~~~~~~~~~~~~~~~~~~~~~~~~~~~~~~~~~~~~~~~~~~~~~~

Type conversions must always be done explicitly. Never rely on
implicit type conversion.

\begin{code}
  floatValue = static_cast<float> (intValue);   // YES! 
  floatValue = intValue;                        // NO! 
\end{code}

\motivation{
  By this, the programmer indicates that he is aware of the different
  types involved and that the mix is intentional.
}

%%%%%%%%%%%%%%%%%%%%%%%%%%%%%%%%%%%%%%%%%%%%%%%%%%%%%%%%%%%%%%%%%%%%%%%%%%%%%%%%
\section{Variables}
%%%%%%%%%%%%%%%%%%%%%%%%%%%%%%%%%%%%%%%%%%%%%%%%%%%%%%%%%%%%%%%%%%%%%%%%%%%%%%%%

%~~~~~~~~~~~~~~~~~~~~~~~~~~~~~~~~~~~~~~~~~~~~~~~~~~~~~~~~~~~~~~~~~~~~~~~~~~~~~~~
\guideline{Variables should be initialized where they are declared}
%~~~~~~~~~~~~~~~~~~~~~~~~~~~~~~~~~~~~~~~~~~~~~~~~~~~~~~~~~~~~~~~~~~~~~~~~~~~~~~~

This ensures that variables are valid at any time. Sometimes it is
impossible to initialize a variable to a valid value where it is
declared:

\begin{code}
  int x, y, z; 
  getCenter (&x, &y, &z); 
\end{code}

In these cases it should be left uninitialized rather than initialized to some phony value.

%~~~~~~~~~~~~~~~~~~~~~~~~~~~~~~~~~~~~~~~~~~~~~~~~~~~~~~~~~~~~~~~~~~~~~~~~~~~~~~~
\guideline{Variables must never have dual meaning}
%~~~~~~~~~~~~~~~~~~~~~~~~~~~~~~~~~~~~~~~~~~~~~~~~~~~~~~~~~~~~~~~~~~~~~~~~~~~~~~~

\motivation{
  Enhance readability by ensuring all concepts are represented
  uniquely. Reduce chance of error by side effects.
}

%~~~~~~~~~~~~~~~~~~~~~~~~~~~~~~~~~~~~~~~~~~~~~~~~~~~~~~~~~~~~~~~~~~~~~~~~~~~~~~~
\guideline{Use of global variables should be minimized}
%~~~~~~~~~~~~~~~~~~~~~~~~~~~~~~~~~~~~~~~~~~~~~~~~~~~~~~~~~~~~~~~~~~~~~~~~~~~~~~~

\motivation{
  In C++ there is no reason global variables need to be used at all. The
  same is true for global functions or file scope (static) variables.
}

%~~~~~~~~~~~~~~~~~~~~~~~~~~~~~~~~~~~~~~~~~~~~~~~~~~~~~~~~~~~~~~~~~~~~~~~~~~~~~~~
\guideline{Non-constant class variables must never be declared public}
%~~~~~~~~~~~~~~~~~~~~~~~~~~~~~~~~~~~~~~~~~~~~~~~~~~~~~~~~~~~~~~~~~~~~~~~~~~~~~~~

\motivation{
  The concept of C++ information hiding and encapsulation is violated by
  public variables. Use private variables and access functions
  instead. One exception to this rule is when the class is essentially a
  data structure, with no behavior (equivalent to a C struct). In this
  case it is appropriate to make the class' instance variables public.
}

Note that structs are kept in C++ for compatibility with C only, and
avoiding them increases the readability of the code by reducing the
number of constructs used. You should use a class instead.

%~~~~~~~~~~~~~~~~~~~~~~~~~~~~~~~~~~~~~~~~~~~~~~~~~~~~~~~~~~~~~~~~~~~~~~~~~~~~~~~
\guideline{Related variables of the same type can be declared in a common statement}
%~~~~~~~~~~~~~~~~~~~~~~~~~~~~~~~~~~~~~~~~~~~~~~~~~~~~~~~~~~~~~~~~~~~~~~~~~~~~~~~

Related variables of the same type can be declared in a common statement.
Unrelated variables should not be declared in the same statement. 

\begin{code}
  float x, y, z;
  float revenueJanuary, revenueFebruary, revenueMarch; 
\end{code}

The common requirement of having declarations on separate lines is not
useful in the situations like the ones above. It enhances readability
to group variables like this. However, in the case the type must not be
a pointer.

Pointers and classes however must be declared on separate lines.

C++ pointers and references should have their reference symbol next to
the type name rather than to the variable name.

\begin{code}
  float* x;    // NOT: float *x;
  int&   y;    // NOT: int &y;
\end{code}

It is debatable whether a pointer is a variable of a pointer type
(float* x) or a pointer to a given type (float *x). It is impossible
to declare more than one pointer in a given statement using the first
approach. I.e. float* x, y, z; is equivalent with float *x; float y;
float z; The same goes for references.

%~~~~~~~~~~~~~~~~~~~~~~~~~~~~~~~~~~~~~~~~~~~~~~~~~~~~~~~~~~~~~~~~~~~~~~~~~~~~~~~
\guideline{\trcode{const} must go after the type}
%~~~~~~~~~~~~~~~~~~~~~~~~~~~~~~~~~~~~~~~~~~~~~~~~~~~~~~~~~~~~~~~~~~~~~~~~~~~~~~~

\begin{code}
  string const& ref = ...;      // NOT: const string& ref
\end{code}

\motivation{
  The type declarations are read from right to left. The type should be the last.
}

%~~~~~~~~~~~~~~~~~~~~~~~~~~~~~~~~~~~~~~~~~~~~~~~~~~~~~~~~~~~~~~~~~~~~~~~~~~~~~~~
\guideline{Constant must be on the left hand side of a comparison}
%~~~~~~~~~~~~~~~~~~~~~~~~~~~~~~~~~~~~~~~~~~~~~~~~~~~~~~~~~~~~~~~~~~~~~~~~~~~~~~~

If a constant is on the left hand side then forgetting a \trcode{=} sign
results in an error message.

\begin{code}
  if (a = 0)        // compiles, but always false
  if (0 = a)        // compiler error
  if (0 == a)       // correct
\end{code}

%~~~~~~~~~~~~~~~~~~~~~~~~~~~~~~~~~~~~~~~~~~~~~~~~~~~~~~~~~~~~~~~~~~~~~~~~~~~~~~~
\guideline{You must not use implicit 0 tests}
%~~~~~~~~~~~~~~~~~~~~~~~~~~~~~~~~~~~~~~~~~~~~~~~~~~~~~~~~~~~~~~~~~~~~~~~~~~~~~~~

Implicit test for 0 should not be used other than for boolean
variables and pointers.

\begin{code}
  if (0 != nLines)        // NOT: if (nLines)
  if (0.0 != value)       // NOT: if (value) 
\end{code}

\motivation{
  It is not necessarily defined by the compiler that ints and floats 0
  are implemented as binary 0. Also, by using explicit test the
  statement give immediate clue of the type being tested.
}

It is common also to suggest that pointers shouldn't test implicit for
0 either, i.e. if (line == 0) instead of if (line). The latter is
regarded as such a common practice in C/C++ however that it can be
used.

%~~~~~~~~~~~~~~~~~~~~~~~~~~~~~~~~~~~~~~~~~~~~~~~~~~~~~~~~~~~~~~~~~~~~~~~~~~~~~~~
\guideline{Variables should be declared in the smallest scope possible. }
%~~~~~~~~~~~~~~~~~~~~~~~~~~~~~~~~~~~~~~~~~~~~~~~~~~~~~~~~~~~~~~~~~~~~~~~~~~~~~~~

\motivation{ 
  Keeping the operations on a variable within a small
  scope, it is easier to control the effects and side effects of the
  variable.
}

%%%%%%%%%%%%%%%%%%%%%%%%%%%%%%%%%%%%%%%%%%%%%%%%%%%%%%%%%%%%%%%%%%%%%%%%%%%%%%%%
\section{Loops}
%%%%%%%%%%%%%%%%%%%%%%%%%%%%%%%%%%%%%%%%%%%%%%%%%%%%%%%%%%%%%%%%%%%%%%%%%%%%%%%%

%~~~~~~~~~~~~~~~~~~~~~~~~~~~~~~~~~~~~~~~~~~~~~~~~~~~~~~~~~~~~~~~~~~~~~~~~~~~~~~~
\guideline{Only loop control statements must be included in the \trcode{for} construction}
%~~~~~~~~~~~~~~~~~~~~~~~~~~~~~~~~~~~~~~~~~~~~~~~~~~~~~~~~~~~~~~~~~~~~~~~~~~~~~~~

\begin{code}
  sum = 0;

  for (i = 0; i < 100; i++) {
    sum += value[i];
  }

  // NOT: for (i = 0, sum = 0; i < 100; i++) {
  //        sum += value[i];
  //      }
\end{code}

\motivation{
  Increase maintainability and readability. Make it crystal clear what
  controls the loop and what the loop contains.
}

%~~~~~~~~~~~~~~~~~~~~~~~~~~~~~~~~~~~~~~~~~~~~~~~~~~~~~~~~~~~~~~~~~~~~~~~~~~~~~~~
\guideline{Should initialise loop variables before the loop}
%~~~~~~~~~~~~~~~~~~~~~~~~~~~~~~~~~~~~~~~~~~~~~~~~~~~~~~~~~~~~~~~~~~~~~~~~~~~~~~~

Loop variables should be initialized immediately before the loop.

\begin{code}
  bool isDone = false;

  while (!isDone) {
    ...
  }

  // NOT: bool isDone = false;
  //      ...
  //      some more code
  ///     ...
  //      while (!isDone) {
  //        ...
  //      } 
\end{code}

%~~~~~~~~~~~~~~~~~~~~~~~~~~~~~~~~~~~~~~~~~~~~~~~~~~~~~~~~~~~~~~~~~~~~~~~~~~~~~~~
\guideline{\trcode{do-while} loops can be avoided}
%~~~~~~~~~~~~~~~~~~~~~~~~~~~~~~~~~~~~~~~~~~~~~~~~~~~~~~~~~~~~~~~~~~~~~~~~~~~~~~~

do-while loops are less readable than ordinary while loops and for
loops since the conditional is at the bottom of the loop. The reader
must scan the entire loop in order to understand the scope of the
loop.  In addition, \trcode{do-while} loops are not needed. Any
\trcode{do-while} loop can easily be rewritten into a \trcode{while}
loop or a \trcode{for} loop. Reducing the number of constructs used
enhance readbility.

%~~~~~~~~~~~~~~~~~~~~~~~~~~~~~~~~~~~~~~~~~~~~~~~~~~~~~~~~~~~~~~~~~~~~~~~~~~~~~~~
\guideline{The use of \trcode{break} and \trcode{continue} in loops should be minimized}
%~~~~~~~~~~~~~~~~~~~~~~~~~~~~~~~~~~~~~~~~~~~~~~~~~~~~~~~~~~~~~~~~~~~~~~~~~~~~~~~

These constructs can be compared to \trcode{goto} and they should only
be used if they prove to have higher readability than their structured
counterpart.

%~~~~~~~~~~~~~~~~~~~~~~~~~~~~~~~~~~~~~~~~~~~~~~~~~~~~~~~~~~~~~~~~~~~~~~~~~~~~~~~
\guideline{The form \trcode{while (true)} should be used for infinite loops}
%~~~~~~~~~~~~~~~~~~~~~~~~~~~~~~~~~~~~~~~~~~~~~~~~~~~~~~~~~~~~~~~~~~~~~~~~~~~~~~~

\begin{code}
  while (true) {
    ... 
  }

  for (;;) { // NO!
    ...
  }

  while (1) { // NO!
    ...
  }
\end{code}

Testing against 1 is neither necessary nor meaningful. The form for
(;;) is not very readable, and it is not apparent that this actually
is an infinite loop.

%%%%%%%%%%%%%%%%%%%%%%%%%%%%%%%%%%%%%%%%%%%%%%%%%%%%%%%%%%%%%%%%%%%%%%%%%%%%%%%%
\section{Conditionals}
%%%%%%%%%%%%%%%%%%%%%%%%%%%%%%%%%%%%%%%%%%%%%%%%%%%%%%%%%%%%%%%%%%%%%%%%%%%%%%%%

%~~~~~~~~~~~~~~~~~~~~~~~~~~~~~~~~~~~~~~~~~~~~~~~~~~~~~~~~~~~~~~~~~~~~~~~~~~~~~~~
\guideline{Complex conditional expressions must be avoided}
%~~~~~~~~~~~~~~~~~~~~~~~~~~~~~~~~~~~~~~~~~~~~~~~~~~~~~~~~~~~~~~~~~~~~~~~~~~~~~~~

Complex conditional expressions must be avoided. Introduce temporary
boolean variables instead.

\begin{code}
 if ((elementNo < 0) || (elementNo > maxElement)||
      elementNo == lastElement) {
    ...
  } 
\end{code}

should be replaced by assuming that short-cutting is not required: 

\begin{code}
  isFinished      = (elementNo < 0) || (elementNo > maxElement);
  isRepeatedEntry = elementNo == lastElement;

  if (isFinished || isRepeatedEntry) {
    ...
  } 
\end{code}

\motivation{
  By assigning boolean variables to expressions, the program gets
  automatic documentation. The construction will be easier to read and
  to debug.
}

%~~~~~~~~~~~~~~~~~~~~~~~~~~~~~~~~~~~~~~~~~~~~~~~~~~~~~~~~~~~~~~~~~~~~~~~~~~~~~~~
\guideline{Should put the exceptional case in the \trcode{else}-part}
%~~~~~~~~~~~~~~~~~~~~~~~~~~~~~~~~~~~~~~~~~~~~~~~~~~~~~~~~~~~~~~~~~~~~~~~~~~~~~~~

The nominal case should be put in the \trcode{if}-part and the exception in the
\trcode{else}-part of an \trcode{if} statement.

\begin{code}
  isError = readFile (fileName);

  if (!isError) {
    ...
  }
  else {
    ...
  } 
\end{code}

\motivation{
  Makes sure that the exceptions don't obscure the normal path of
  execution. This is important for both the readability and performance.
}

%~~~~~~~~~~~~~~~~~~~~~~~~~~~~~~~~~~~~~~~~~~~~~~~~~~~~~~~~~~~~~~~~~~~~~~~~~~~~~~~
\guideline{The conditional should be put on a separate line}
%~~~~~~~~~~~~~~~~~~~~~~~~~~~~~~~~~~~~~~~~~~~~~~~~~~~~~~~~~~~~~~~~~~~~~~~~~~~~~~~

\begin{code}
  if (isDone) {            // NOT: if (isDone) { doCleanup(); }
    doCleanup();
  } 
\end{code}

\motivation{
  This is for debugging purposes. When writing on a single line, it is
  not apparent whether the test is really true or not.
}

%~~~~~~~~~~~~~~~~~~~~~~~~~~~~~~~~~~~~~~~~~~~~~~~~~~~~~~~~~~~~~~~~~~~~~~~~~~~~~~~
\guideline{Executable statements in conditionals must be avoided}
%~~~~~~~~~~~~~~~~~~~~~~~~~~~~~~~~~~~~~~~~~~~~~~~~~~~~~~~~~~~~~~~~~~~~~~~~~~~~~~~

\begin{code}
  // Bad!
  if (!(fileHandle = open (fileName, "w"))) { 
    ... 
  } 

  // Better!
  fileHandle = open (fileName, "w");

  if (!fileHandle) {
    ...
  }
\end{code}

\motivation{ 
  Conditionals with executable statements are just very
  difficult to read. This is especially true for programmers new to
  C/C++.
}

%%%%%%%%%%%%%%%%%%%%%%%%%%%%%%%%%%%%%%%%%%%%%%%%%%%%%%%%%%%%%%%%%%%%%%%%%%%%%%%%
\section{Miscellaneous}
%%%%%%%%%%%%%%%%%%%%%%%%%%%%%%%%%%%%%%%%%%%%%%%%%%%%%%%%%%%%%%%%%%%%%%%%%%%%%%%%

%~~~~~~~~~~~~~~~~~~~~~~~~~~~~~~~~~~~~~~~~~~~~~~~~~~~~~~~~~~~~~~~~~~~~~~~~~~~~~~~
\guideline{\trcode{goto} should not be used}
%~~~~~~~~~~~~~~~~~~~~~~~~~~~~~~~~~~~~~~~~~~~~~~~~~~~~~~~~~~~~~~~~~~~~~~~~~~~~~~~

\motivation{ \trcode{goto} statements violates the idea of structured
  code. Only in some very few cases (for instance breaking out of
  deeply nested structures or for special needs when dealing with an
  error) should \trcode{goto} be considered, and only if the
  alternative structured counterpart (for example, exceptions) is
  proven to be less readable.
}

%~~~~~~~~~~~~~~~~~~~~~~~~~~~~~~~~~~~~~~~~~~~~~~~~~~~~~~~~~~~~~~~~~~~~~~~~~~~~~~~
\guideline{Functions must always have the return value explicitly listed}
%~~~~~~~~~~~~~~~~~~~~~~~~~~~~~~~~~~~~~~~~~~~~~~~~~~~~~~~~~~~~~~~~~~~~~~~~~~~~~~~

\begin{code}
 int getValue() { // NOT: getValue()
   ...
 }
\end{code}

\motivation{
  If not explicitly listed, C++ implies int return value for functions. A
  programmer must never rely on this feature, since this might be
  confusing for programmers not aware of this artifact.
}

%~~~~~~~~~~~~~~~~~~~~~~~~~~~~~~~~~~~~~~~~~~~~~~~~~~~~~~~~~~~~~~~~~~~~~~~~~~~~~~~
\guideline{\trcode{0} should be used instead of \trcode{NULL} in C++}
%~~~~~~~~~~~~~~~~~~~~~~~~~~~~~~~~~~~~~~~~~~~~~~~~~~~~~~~~~~~~~~~~~~~~~~~~~~~~~~~

\motivation{
  NULL is part of the standard C library, but is made obsolete in C++. 
}

%%%%%%%%%%%%%%%%%%%%%%%%%%%%%%%%%%%%%%%%%%%%%%%%%%%%%%%%%%%%%%%%%%%%%%%%%%%%%%%%
\chapter{Miscellaneous}
%%%%%%%%%%%%%%%%%%%%%%%%%%%%%%%%%%%%%%%%%%%%%%%%%%%%%%%%%%%%%%%%%%%%%%%%%%%%%%%%

%%%%%%%%%%%%%%%%%%%%%%%%%%%%%%%%%%%%%%%%%%%%%%%%%%%%%%%%%%%%%%%%%%%%%%%%%%%%%%%%
\section{Miscellaneous}
%%%%%%%%%%%%%%%%%%%%%%%%%%%%%%%%%%%%%%%%%%%%%%%%%%%%%%%%%%%%%%%%%%%%%%%%%%%%%%%%

%~~~~~~~~~~~~~~~~~~~~~~~~~~~~~~~~~~~~~~~~~~~~~~~~~~~~~~~~~~~~~~~~~~~~~~~~~~~~~~~
\guideline{The use of magic numbers in the code must be avoided}
%~~~~~~~~~~~~~~~~~~~~~~~~~~~~~~~~~~~~~~~~~~~~~~~~~~~~~~~~~~~~~~~~~~~~~~~~~~~~~~~

Numbers other than 0 and 1 must be declared as named constants instead.

\motivation{
  If the number does not have an obvious meaning by itself, the
  readability is enhanced by introducing a named constant instead. A
  different approach is to introduce a method from which the constant
  can be accessed.
}

%~~~~~~~~~~~~~~~~~~~~~~~~~~~~~~~~~~~~~~~~~~~~~~~~~~~~~~~~~~~~~~~~~~~~~~~~~~~~~~~
\guideline{The use of magic strings in the code should be avoided}
%~~~~~~~~~~~~~~~~~~~~~~~~~~~~~~~~~~~~~~~~~~~~~~~~~~~~~~~~~~~~~~~~~~~~~~~~~~~~~~~

Strings of length 3 or greater should be considered declared as named
constants instead.

\motivation{
  Using a constant reduces the errors due to typing mistakes. Strings used in
  text messages should only be declared as constants if used more than once.
}

%~~~~~~~~~~~~~~~~~~~~~~~~~~~~~~~~~~~~~~~~~~~~~~~~~~~~~~~~~~~~~~~~~~~~~~~~~~~~~~~
\guideline{Use a decimal point for floating point constants}
%~~~~~~~~~~~~~~~~~~~~~~~~~~~~~~~~~~~~~~~~~~~~~~~~~~~~~~~~~~~~~~~~~~~~~~~~~~~~~~~

Floating point constants should always be written with decimal point
and at least one decimal.

\begin{code}
  double total = 0.0;     // NOT: double total = 0; 
  double speed = 3.0e8;   // NOT: double speed = 3e8;
  double sum;
  ...
  sum = (a + b) * 10.0; 
\end{code}

\motivation{
  This emphasize the different nature of integer and floating point
  numbers even if their values might happen to be the same in a specific
  case.
}

Also, as in the last example above, it emphasize the type of the
assigned variable (sum) at a point in the code where this might not be
evident.

%~~~~~~~~~~~~~~~~~~~~~~~~~~~~~~~~~~~~~~~~~~~~~~~~~~~~~~~~~~~~~~~~~~~~~~~~~~~~~~~
\guideline{Floating point constants should always be written with a digit before the decimal point}
%~~~~~~~~~~~~~~~~~~~~~~~~~~~~~~~~~~~~~~~~~~~~~~~~~~~~~~~~~~~~~~~~~~~~~~~~~~~~~~~

\begin{code}
  double total = 0.5;             // NOT: double total = .5; 
\end{code}

\motivation{
  The number and expression system in C++ is borrowed from mathematics
  and one should adhere to mathematical conventions for syntax wherever
  possible. Also, 0.5 is a lot more readable than .5; There is no way it
  can be mixed with the integer 5.
}

%~~~~~~~~~~~~~~~~~~~~~~~~~~~~~~~~~~~~~~~~~~~~~~~~~~~~~~~~~~~~~~~~~~~~~~~~~~~~~~~
\guideline{Floating point comparison should be avoid}
%~~~~~~~~~~~~~~~~~~~~~~~~~~~~~~~~~~~~~~~~~~~~~~~~~~~~~~~~~~~~~~~~~~~~~~~~~~~~~~~

Floating point comparison is dangerous due to rounding errors. It should therefore
be handled with care.

%%%%%%%%%%%%%%%%%%%%%%%%%%%%%%%%%%%%%%%%%%%%%%%%%%%%%%%%%%%%%%%%%%%%%%%%%%%%%%%%
\chapter{Layout and Comments}
%%%%%%%%%%%%%%%%%%%%%%%%%%%%%%%%%%%%%%%%%%%%%%%%%%%%%%%%%%%%%%%%%%%%%%%%%%%%%%%%

%%%%%%%%%%%%%%%%%%%%%%%%%%%%%%%%%%%%%%%%%%%%%%%%%%%%%%%%%%%%%%%%%%%%%%%%%%%%%%%%
\section{Layout}
%%%%%%%%%%%%%%%%%%%%%%%%%%%%%%%%%%%%%%%%%%%%%%%%%%%%%%%%%%%%%%%%%%%%%%%%%%%%%%%%

%~~~~~~~~~~~~~~~~~~~~~~~~~~~~~~~~~~~~~~~~~~~~~~~~~~~~~~~~~~~~~~~~~~~~~~~~~~~~~~~
\guideline{Tabs must be eight characters wide}
%~~~~~~~~~~~~~~~~~~~~~~~~~~~~~~~~~~~~~~~~~~~~~~~~~~~~~~~~~~~~~~~~~~~~~~~~~~~~~~~

\motivation{
  Since the beginning of time.
}

%~~~~~~~~~~~~~~~~~~~~~~~~~~~~~~~~~~~~~~~~~~~~~~~~~~~~~~~~~~~~~~~~~~~~~~~~~~~~~~~
\guideline{Basic indentation should be 2}
%~~~~~~~~~~~~~~~~~~~~~~~~~~~~~~~~~~~~~~~~~~~~~~~~~~~~~~~~~~~~~~~~~~~~~~~~~~~~~~~

\begin{code}
  for (i = 0; i < nElements; i++) {
    a[i] = 0; 
  }
\end{code}

\motivation{ 
  Indentation of 1 is to small to emphasize the logical
  layout of the code. Indentation larger than 4 makes deeply nested
  code difficult to read and increase the chance that the lines must
  be split. Choosing between indentation of 2, 3 and 4, 2 and 4 are
  the more common, and 2 chosen to reduce the chance of splitting code
  lines.
}

Block layout should be as illustrated in example 1 below (recommended)
or example 2, and must not be as shown in example 3. Function and
class blocks must use the block layout of example 1.

\begin{code}
  // example 1
  while (!done) {
    doSomething();
    done = moreToDo();
  }

  // example 2
  while (!done)
  {
    doSomething();
    done = moreToDo();
  }

  // example 3
  while (!done)
    {
      doSomething();
      done = moreToDo();
    }
\end{code}

Example 3 introduce an extra indentation level which doesn't emphasize
the logical structure of the code as clearly as example 1 and 2.

The class declarations should have the following form: 

\begin{code}
  class SomeClass : public BaseClass {
    public: 
      void doSomething ();

    protected: 
      ... 

    private: 
      ... 
  }
\end{code}

This follows partly from the general block rule above.

The function declarations should have the following form: 

\begin{code}
  void someMethod () {
    ...
  }
\end{code}

This follows from the general block rule above. Note that there is an
extra space before the \trcode{()}. When calling a function there
should be no space.

The \trcode{if-else} statements should have the following form: 

\begin{code}
  if (condition) {
    statements;
  }

  if (condition) {
    statements;
  }
  else {
    statements;
  }

  if (condition) {
    statements;
  }
  else if (condition) {
    statements;
  }
  else {
    statements;
  }
\end{code}

This follows partly from the general block rule above. An else clause should not
be on the same line as the closing bracket of the previous if or else clause:

\begin{code}
  if (condition) {
    statements;
  } else {        // NO
    statements;
  }
\end{code}

\motivation{
  This is equivalent to the Sun recommendation. The chosen approach is
  considered better in the way that each part of the \trcode{if-else} statement
  is written on separate lines of the file. This should make it easier
  to manipulate the statement, for instance when moving else clauses
  around.
}

A for statement should have the following form: 

\begin{code}
  for (initialization; condition; update) {
    statements;
  }
\end{code}

This follows from the general block rule above.

An empty for  statement should have the following form: 

\begin{code}
  for (initialization; condition; update) {
  }
\end{code}

\motivation{
  This emphasize the fact that the for statement is empty and it makes
  it obvious for the reader that this is intentional. Empty loops should
  be avoided however.
}

A while  statement should have the following form: 

\begin{code}
  while (condition) {
    statements;
  }
\end{code}

This follows from the general block rule above.  

A do-while  statement should have the following form: 

\begin{code}
  do {
    statements;
  } while (condition);
\end{code}

This follows from the general block rule above.

A switch  statement should have the following form: 

\begin{code}
  switch (condition) {
    case ABC: 
      statements;
      // Fallthrough

    case DEF: 
      statements;
      break;

    case XYZ1: 
    case XYZ2: 
      statements;
      break;

    default: 
      statements;
      break;
  }
\end{code}

Note that each \trcode{case} keyword is indented relative to the switch
statement as a whole. Note also that no extra space before the : character
exits. The explicit \trcode{// Fallthrough} comment must be included whenever
there is a case statement without a \trcode{break} statement. Leaving the
\trcode{break} out is a common error, and it must be made clear that it is
intentional when it is not there.

A \trcode{try-catch} statement should have the following form: 

\begin{code}
  try {
    statements;
  }
  catch (Exception &exception) {
    statements;
  }
\end{code}

This follows partly from the general block rule above. The discussion about
closing brackets for \trcode{if-else} statements apply to the \trcode{try-catch}
statments.

The function return type must be put in the immediately before the function
name.

\begin{code}
  void MyClass::myMethod (void) {
    ...
  } 
\end{code}

%~~~~~~~~~~~~~~~~~~~~~~~~~~~~~~~~~~~~~~~~~~~~~~~~~~~~~~~~~~~~~~~~~~~~~~~~~~~~~~~
\guideline{Fallthrough comment must be used}
%~~~~~~~~~~~~~~~~~~~~~~~~~~~~~~~~~~~~~~~~~~~~~~~~~~~~~~~~~~~~~~~~~~~~~~~~~~~~~~~

See the example above.

%~~~~~~~~~~~~~~~~~~~~~~~~~~~~~~~~~~~~~~~~~~~~~~~~~~~~~~~~~~~~~~~~~~~~~~~~~~~~~~~
\guideline{Empty blocks or on-line blocks must use brackets}
%~~~~~~~~~~~~~~~~~~~~~~~~~~~~~~~~~~~~~~~~~~~~~~~~~~~~~~~~~~~~~~~~~~~~~~~~~~~~~~~

Even if a block contains only a single statement, it must be enclosed in
brackets.

\begin{code}
  if (0 == a) {
    doIt();
  }

  if (1 == a) {
    // do nothing because ...
  }
\end{code}

\motivation{
   If a if-clause starts out is one-line and is later extended it is easy to forget
   the brackets.

  It is a common recommendation (Sun Java recommendation included) that
  brackets should always be used in all these cases.
}

%%%%%%%%%%%%%%%%%%%%%%%%%%%%%%%%%%%%%%%%%%%%%%%%%%%%%%%%%%%%%%%%%%%%%%%%%%%%%%%%
\section{White Space}
%%%%%%%%%%%%%%%%%%%%%%%%%%%%%%%%%%%%%%%%%%%%%%%%%%%%%%%%%%%%%%%%%%%%%%%%%%%%%%%%

%~~~~~~~~~~~~~~~~~~~~~~~~~~~~~~~~~~~~~~~~~~~~~~~~~~~~~~~~~~~~~~~~~~~~~~~~~~~~~~~
\guideline{Conventional operators should be surrounded by a space character}
%~~~~~~~~~~~~~~~~~~~~~~~~~~~~~~~~~~~~~~~~~~~~~~~~~~~~~~~~~~~~~~~~~~~~~~~~~~~~~~~

C++ reserved words should be followed by a white space.  Commas should be
followed by a white space.  Colons should be surrounded by white space.  Colons
in \trcode{case} statements should not be surrounded by white space.  Semicolons
in \trcode{for} statments should be followed by one or two (recommended) space
characters.

\begin{code}
  a = (b + c) * d;           // NOT: a=(b+c)*d

  while (true) {             // NOT: while(true) ...

  doSomething(a, b, c, d);   // NOT: doSomething(a,b,c,d);

  case 100:                  // NOT: case 100 :

  for (i = 0;  i < 10;  i++) { // NOT: for (i=0;i<10;i++){ 
\end{code}

\motivation{ 
  Makes the individual components of the statements stand
  out. Enhances readability. It is difficult to give a complete list
  of the suggested use of whitespace in C++ code. The examples above
  however should give a general idea of the intentions.
}

%~~~~~~~~~~~~~~~~~~~~~~~~~~~~~~~~~~~~~~~~~~~~~~~~~~~~~~~~~~~~~~~~~~~~~~~~~~~~~~~
\guideline{The method name should be followed by a space}
%~~~~~~~~~~~~~~~~~~~~~~~~~~~~~~~~~~~~~~~~~~~~~~~~~~~~~~~~~~~~~~~~~~~~~~~~~~~~~~~

Method or function names should be followed by a white space in the declaration
or implementation, but not when calling.

\begin{code}
  void doSomething (FILE *currentFile) {
  }

  doSomething(currentFile); 
\end{code}

\motivation{ 
  Allows to search for \trcode{doSomething (} to find the
  declaration or implementation.
}

%~~~~~~~~~~~~~~~~~~~~~~~~~~~~~~~~~~~~~~~~~~~~~~~~~~~~~~~~~~~~~~~~~~~~~~~~~~~~~~~
\guideline{Logical units within a block should be separated by one blank line}
%~~~~~~~~~~~~~~~~~~~~~~~~~~~~~~~~~~~~~~~~~~~~~~~~~~~~~~~~~~~~~~~~~~~~~~~~~~~~~~~

A blank line should be inserted before and after a \trcode{if} or \trcode{while}
block.

\begin{code}
  int f = 1;

  if (f > e) {
    if (g > h) {
      int a = f + e + g + h;

      if (a == 1) {
        ...
      }
    }

    int x = f + 1;

    ...
  }
\end{code}

\motivation{
  Enhance readability by introducing white space between logical units of a block.
}

%~~~~~~~~~~~~~~~~~~~~~~~~~~~~~~~~~~~~~~~~~~~~~~~~~~~~~~~~~~~~~~~~~~~~~~~~~~~~~~~
\guideline{Methods should be separated by blank lines}
%~~~~~~~~~~~~~~~~~~~~~~~~~~~~~~~~~~~~~~~~~~~~~~~~~~~~~~~~~~~~~~~~~~~~~~~~~~~~~~~

They should be separated with three blank lines in larger files.

\motivation{
  The methods will stand out within the file.
}

%~~~~~~~~~~~~~~~~~~~~~~~~~~~~~~~~~~~~~~~~~~~~~~~~~~~~~~~~~~~~~~~~~~~~~~~~~~~~~~~
\guideline{Variables in declarations can be left aligned}
%~~~~~~~~~~~~~~~~~~~~~~~~~~~~~~~~~~~~~~~~~~~~~~~~~~~~~~~~~~~~~~~~~~~~~~~~~~~~~~~

\begin{code}
  AsciiFile *  file;
  int          nPoints;
  float        x, y;
\end{code}

\motivation{
  Enhance readability. The variables are easier to spot from the types by alignment.
  Use alignment wherever it enhances readability. 
}

\begin{code}
  if      (a == lowValue)    computeSomething();
  else if (a == mediumValue) computeSomethingElse();
  else if (a == highValue)   computeSomethingElseYet();

  value = (potential        * oilDensity)   / constant1 +
          (depth            * waterDensity) / constant2 +
          (zCoordinateValue * gasDensity)   / constant3;

  minPosition     = computeDistance (min,     x, y, z);
  averagePosition = computeDistance (average, x, y, z);

  switch (value) {
    case PHASE_OIL:   strcpy (string, "Oil");   break;
    case PHASE_WATER: strcpy (string, "Water"); break;
    case PHASE_GAS:   strcpy (string, "Gas");   break;
  } 
\end{code}

\motivation{
  There are a number of places in the code where white space can be
  included to enhance readability even if this violates common
  guidelines. Many of these cases have to do with code
  alignment. General guidelines on code alignment are difficult to give,
  but the examples above should give a general clue.
}

%%%%%%%%%%%%%%%%%%%%%%%%%%%%%%%%%%%%%%%%%%%%%%%%%%%%%%%%%%%%%%%%%%%%%%%%%%%%%%%%
\section{Comments}
%%%%%%%%%%%%%%%%%%%%%%%%%%%%%%%%%%%%%%%%%%%%%%%%%%%%%%%%%%%%%%%%%%%%%%%%%%%%%%%%

%~~~~~~~~~~~~~~~~~~~~~~~~~~~~~~~~~~~~~~~~~~~~~~~~~~~~~~~~~~~~~~~~~~~~~~~~~~~~~~~
\guideline{Tricky code should not be commented but rewritten}
%~~~~~~~~~~~~~~~~~~~~~~~~~~~~~~~~~~~~~~~~~~~~~~~~~~~~~~~~~~~~~~~~~~~~~~~~~~~~~~~

\motivation{
  In general, the use of comments should be minimized by making the code
  self-documenting by appropriate name choices and an explicit logical
  structure.
}

%~~~~~~~~~~~~~~~~~~~~~~~~~~~~~~~~~~~~~~~~~~~~~~~~~~~~~~~~~~~~~~~~~~~~~~~~~~~~~~~
\guideline{All comments should be written in english}
%~~~~~~~~~~~~~~~~~~~~~~~~~~~~~~~~~~~~~~~~~~~~~~~~~~~~~~~~~~~~~~~~~~~~~~~~~~~~~~~

\motivation{
  In an international environment english is the preferred language.
}

%~~~~~~~~~~~~~~~~~~~~~~~~~~~~~~~~~~~~~~~~~~~~~~~~~~~~~~~~~~~~~~~~~~~~~~~~~~~~~~~
\guideline{Use \trcode{//} for all comments, including multi-line comments}
%~~~~~~~~~~~~~~~~~~~~~~~~~~~~~~~~~~~~~~~~~~~~~~~~~~~~~~~~~~~~~~~~~~~~~~~~~~~~~~~

\begin{code}
  // Comment spanning
  // more than one line. 
\end{code}

\motivation{
  Since multilevel C-commenting is not supported, using \trcode{//} comments
  ensure that it is always possible to comment out entire sections of a
  file using \trcode{/* */} for debugging purposes etc.
}

There should be a space between the \trcode{//} and the actual comment.
Comments should be included relative to their position in the code.

\begin{code}
  while (true) {          // NOT:    while (true) {
    // Do something       //         // Do something
    something();          //           something();
  }                       //         } 
\end{code}

\motivation{
  This is to avoid that the comments break the logical structure of the program.  
}

%~~~~~~~~~~~~~~~~~~~~~~~~~~~~~~~~~~~~~~~~~~~~~~~~~~~~~~~~~~~~~~~~~~~~~~~~~~~~~~~
\guideline{Must use development comments only during development}
%~~~~~~~~~~~~~~~~~~~~~~~~~~~~~~~~~~~~~~~~~~~~~~~~~~~~~~~~~~~~~~~~~~~~~~~~~~~~~~~

The comments \trcode{FIXME} and \trcode{TODO} must be used to flag code blocks
which must be refactored. The release code must be free of such comments and
code blocks.

%~~~~~~~~~~~~~~~~~~~~~~~~~~~~~~~~~~~~~~~~~~~~~~~~~~~~~~~~~~~~~~~~~~~~~~~~~~~~~~~
\guideline{You must use doxygen conventions for C++}
%~~~~~~~~~~~~~~~~~~~~~~~~~~~~~~~~~~~~~~~~~~~~~~~~~~~~~~~~~~~~~~~~~~~~~~~~~~~~~~~

Class and method header comments must follow the Doxygen conventions.
 
\begin{code}
  ////////////////////////////////////////////////////////////////////////////////
  /// @brief returns the maximum of two numbers
  ////////////////////////////////////////////////////////////////////////////////
\end{code}

\motivation{
  Regarding standardized class and method documentation the Java development
  community is far more mature than the C++. This is of course because Java
  includes a tool for extracting such comments and produce high quality
  hypertext documentation from it.  There have never been a common convention
  for writing this kind of documentation in C++, so when choosing between
  inventing your own convention, and using an existing one, the latter option
  seem natural. Also, there are JavaDoc tools for C++ available. See for
  instance Doc++ or Doxygen.
}

For inherited methods use

\begin{code}
  ////////////////////////////////////////////////////////////////////////////////
  /// {@inheritDoc}
  ////////////////////////////////////////////////////////////////////////////////
\end{code}

%~~~~~~~~~~~~~~~~~~~~~~~~~~~~~~~~~~~~~~~~~~~~~~~~~~~~~~~~~~~~~~~~~~~~~~~~~~~~~~~
\guideline{Sectionise large files}
%~~~~~~~~~~~~~~~~~~~~~~~~~~~~~~~~~~~~~~~~~~~~~~~~~~~~~~~~~~~~~~~~~~~~~~~~~~~~~~~

Use section descriptions to split large files into sections.

\begin{code}
  // /////////////////////////////////////////////////////////////////////////////
  // constructors and destructors
  // /////////////////////////////////////////////////////////////////////////////

  ...

  // /////////////////////////////////////////////////////////////////////////////
  // public methods
  // /////////////////////////////////////////////////////////////////////////////

  ...
\end{code}
 
%%%%%%%%%%%%%%%%%%%%%%%%%%%%%%%%%%%%%%%%%%%%%%%%%%%%%%%%%%%%%%%%%%%%%%%%%%%%%%%%
\chapter{References}
%%%%%%%%%%%%%%%%%%%%%%%%%%%%%%%%%%%%%%%%%%%%%%%%%%%%%%%%%%%%%%%%%%%%%%%%%%%%%%%%
\label{cha:references}

\begin{enumerate}
  \item
    Code Complete, \\
    Steve McConnel - Microsoft Press
    \label{bib:1}

  \item
    Programming in C++, Rules and Recommendations, \\
    M Henricson, e. Nyquist, Ellemtel (Swedish telecom), \\
    http://www.doc.ic.ac.uk/lab/cplus/c%2b%2b.rules/
    \label{bib:2}

  \item
    Wildfire C++ Programming Style, \\
    Keith Gabryelski, Wildfire Communications Inc., \\
    http://www.wildfire.com/~ag/Engineering/Development/C++Style/
    \label{bib:3}

  \item
    C++ Coding Standard, \\
    Todd Hoff, \\
    http://www.possibility.com/Cpp/CppCodingStandard.htm 
    \label{bib:4}

  \item
    Doxygen documentation system, \\
    http://www.stack.nl/~dimitri/doxygen/index.html
    \label{bib:5}
\end{enumerate}
%
\end{document}
%
%%% Local Variables: 
%%% mode: latex
%%% TeX-master: t
%%% End: 
